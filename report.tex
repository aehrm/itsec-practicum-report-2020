\documentclass[10pt,a4paper,twocolumn]{article}

\usepackage{unicode-math}
\setmainfont{Libertinus Serif}
\setmathfont{Libertinus Math}

\usepackage{polyglossia}
\setdefaultlanguage{english}

\usepackage{booktabs}
\usepackage[binary-units=true]{siunitx}
\usepackage{csquotes}
\usepackage{enumitem}
\setlist{itemsep=0.3ex}
\setlist[enumerate]{label = (\alph*)}
\usepackage[backend=biber,style=alphabetic]{biblatex}

\title{Evaluation of countermeasures against malicious data inclusion attacks in Bitcoin}
\author{Anton Ehrmanntraut}

\renewenvironment{abstract}
{\begin{quote}
\noindent {\bfseries \abstractname.}}
{\end{quote}
}

\begin{document}

\maketitle
\begin{abstract}
\end{abstract}

\section{Introduction}

\section{Motivation and Scope}

% ignore input-script-insertions <-> assume non-standard scripts banned
% P2SH-method uses non-standard script and is not suited for analysis(?)
% required unlocking input however provably is a  public key

\section{Approach}

\section{Increased difficulty of data inclusion}

% TODO intros

\subsection{Modelling inclusion cost}

Essentially, the choice of a suitable prefix length is a trade-off:
On the one hand, the number of required keypairs, and hence needed transaction fees, are inversely proportional to the chosen prefix length;
on the other hand, the probability for a partial collision decreases exponentially with respect to the prefix length, and in turn, computation cost increases exponentially.

To precicely model the cost of an inclusion of an arbitrary payload into the blockchain, restricted with the proposed countermeasure, let us denote 
\begin{itemize}
    \item prefix length with $n$,
    \item payload length with $N$,
    \item cost of a single hash operation with $c$,
    \item transaction fee for a single output with $f$.
\end{itemize}
Moreover, we denote $p=N/n$ as the number of required outputs, (or equivalently, the number of payload fragments) and $X$ as the random variable of required hash operations.
From this, total cost $C$ of an inclusion can be given with
\[ C =  c X + fp = c X + fN/n . \]

To precicely determine random variable $X$, we remind ourselves of the general iterative procedure of the brute-force algorithm:
In each iteration, the algorithm determines a keypair such that its prefix matches one of the payload fragments not yet assigned a keypair, and then assigns that fragment the keypair.
That is, in iteration $i$, the algorithm chooses private keys randomly, until generated public key paritially collides with one of the $p-i$ remaining unassigned payload fragments.
Under the worst case, all fragments are pairwise different%
%\footnote{This is to be expected for sufficiently large prefix length $n$:}%
, hence one iteration can be viewed as repeated independent Bernoulli trials until success (i.e.\@ partial collision found), with success probability $(p-i)/2^n$.  
Refering to required hashes in iteration $i$ with $X_i$, we thus observe that the random variable $X_i$ follows a geometric distribution with parameter $(p-i)/2^n$.
For the total number of required hash operations $X$, we reach
\[ X = \sum_{i=1}^{p} X_i, \quad E[X] = \sum_{i=1}^{p} E[X_i] = \sum_{i=1}^{p}\frac{2^n}{i} = \frac{2\cdot 2^{n}}{p^2+p}, \]
and conclude for the expected total cost $C$
\begin{equation}
    E[C] = \frac{2c2^{n}}{p^2+p} + fp.\label{eq:totalcost}
\end{equation}

Figure \ref{fig:xxx} plots the expected total cost with respect to chosen prefix length, using $N=\SI{1000}{\byte}$, and sensible values for $f$, $c$, assuming the {P2PK} method (cf. next section).
We can clearly observe a minimum value, the inverse proportionality in bit ranges left of highlighted minimum value, and the exponential growth right of the minimum value.

\subsection{Estimation of parameters}

% TODO intros

The software introduced in previous section was used to measure values for $c$, separately for each inclusion method ({P2PK, P2PKH, P2SH}).
% TODO Hardware
Additionally, we can give further estimates for cost $c$ indirectly for the inclusion method {P2PKH} and {P2SH}.

\paragraph{Estimation using \emph{Vanitygen}}
For the {P2PKH} method, we can additionally rely on user reports on their performance using the tool \emph{Vanitygen}, which allows users to create vanity addresses.
These Bitcoin addresses contain a human-readable prefix in their Base-58 representation.
Similiarly to the presented tool in previous section, \emph{Vanitygen} brute-forces private keys, until the hashed public key has the desired prefix;
this public key hash is preciely the one used to create {P2PKH} outputs.

Hence, due to the computational similarity, it seems reasonable to estimate possible computational cost for the {P2PKH} method from reported key frequencies of \emph{Vanitygen} on selected hardware.
In fact, the \emph{OpenCL} code used for the (usually faster performing) GPU code direcly builds on top of the one used in \emph{Vanitygen}.
Table \ref{table:vanitygen} gives estimations for cost $c$ bases on selected user reports.
% TODO comparison

\begin{table*}
    \centering
    \begin{tabular}{lr<{\,\si{\watt}}r<{\,\si{\mega\hertz}}S[table-format=2.1e2]r}
        \toprule
        Hardware & \multicolumn{1}{r}{est. wattage} &\multicolumn{1}{r}{reported freq.} & \multicolumn{1}{r}{$c$ in USD} & ref.\cr
        %\midrule
        % TODO personal measures
        \midrule
        AMD RX 480  & 225 & 60 & 1.4e-13 &  Aug. 2016\cr %https://en.bitcoin.it/w/index.php?title=Vanitygen&type=revision&diff=61424&oldid=58554
        GTX 1060  & 120 & 40 & 1.1e-13 &  Mar. 2018\cr
        GTX 1080 TI  & 250 & 100 & 9.0e-14 &  Feb. 2019\cr
        \bottomrule
    \end{tabular}
    \caption{User's reports of their brute-force frequencies for \emph{Vanitygen} on specific hardware. We estimate cost parameter $c$ for the {P2PKH} by first researching estimated power consumption of the hardware under full load, and assuming energy cost of \mbox{\SI{0.13}{\$/\kilo\watt\hour}}.}
    \label{table:vanitygen}
\end{table*}

\paragraph{Estimation based on password crackers}
In a similar matter, one can use the performances in computing {SHA256} digests as estimation for the cost of the {P2SH} method.
%
%In a similar matter, one can use Bitcoin mining hardware as comparison for cost of the {P2SH} method.
%We argue that the computational effort required in the Proof-of-Work mining process is comparable to the one required in hashing a {P2SH} output.
%
As was outlined in previous section, the brute-force procedure for the {P2SH} method constructs from nonce candidate $x$ a 57-byte long redeem script $S(x)$, and computes the {P2SH} output address as the hash digest 
\begin{equation}
    x \mapsto \text{{RIPEMD160}}(\text{{SHA256}}(S(x))).\label{eq:p2sh-hash}
\end{equation}
% http://bench.cr.yp.to/results-hash.html#amd64-skylake
% https://en.wikipedia.org/wiki/Comparison_of_cryptographic_hash_functions
% KL15
The primitive hash functions {RIPEMD160} and {SHA256} are highly similar in architecture and performance; their respective Merkle–Damgård transforms also operate on the same block size of 512 bits.
Therefore, the effort required in above procedure (\ref{eq:p2sh-hash}) can be considered twice as large as one {SHA256} digest computation (in the sense of input length $\leq\!\SI{512}{\bit}$). 

% hashcat

% TODO move to sec. 5

Only as a theoretical consideration, we can build an additional estimation on the basis of Bitcoin mining hardware.
The emergence of mining hardware developed using application-specific integrated circuits (ASIC) proved that SHA-256 hashes can be computed with an efficiency magnitudes better than traditional GPU-based methods.
Nevertheless, reliable numbers for the efficiency of such ASIC units seem to only appear in the context of Bitcoin mining, therefore, we focus on such mining hardware.
It has to be emphasized that the advertized strong computation power of such ASIC units cannot directly be utilized for the task of performing the P2SH computation (\ref{eq:p2sh-hash}), and the determined numbers only serve as an indirect estimation of the theoretical efficiency permitted by application-specific integrated circuits.

In comparison to the P2SH hash procedure (\ref{eq:p2sh-hash}), Proof-of-Work mining constructs from nonce candidate $x$ the 80-byte long block header $H(x)$, and performs a double SHA-256 hashing, i.e. 
\begin{equation}
    x \mapsto \text{{SHA256}}(\text{{SHA256}}(H(x))),\label{eq:pow-hash}
\end{equation}
to test against network target.


% TODO
This conclusion is not immediate due to following apparent caveat: block headers $H(x)$ are substantially longer than redeem scripts $S(x)$.
As the former input is split into two 512-bit blocks during hasing, computation requires two invocations of the respective {SHA256} compression function in the Merkle–Damgård transform.
However, the first 512 bits of input $H(x)$ remain constant with respect to chosen nonce $x$, and in general, during mining, the \enquote*{mid-state} after applying the first compression is precomputed and stored.
Hence, computations of $\text{{SHA256}}(S(x))$ and $\text{{SHA256}}(H(x))$ both require only a single invocation of the compression function.
% https://link.springer.com/chapter/10.1007/978-3-662-44893-9_12

% TODO mining process
%ASIC mining hardware manufacturers provide performance measures for their advertised hardware, 

\begin{table*}
    \centering
    \begin{tabular}{lr<{\,\si{\watt}}S[table-number-alignment=right,table-column-width=1.4cm]@{}rS[table-format=1.1e2]r}
        \toprule
        Hardware & \multicolumn{1}{r}{est. wattage} &\multicolumn{2}{r}{reported freq.} & \multicolumn{1}{r}{$c$ in USD} & ref.\cr
        \midrule
        GTX 1080 TI & 250 & 2.4 & \si{\giga\hertz} & 3.8e-15 & \cr % https://gist.github.com/epixoip/ace60d09981be09544fdd35005051505
        GTX 2080 TI & 280 & 3.8 & \si{\giga\hertz} & 2.7e-15 & \cr % https://gist.github.com/epixoip/ace60d09981be09544fdd35005051505
        %\midrule
        %Antminer S9* & 1375 & 14 & \si{\tera\hertz} & 3.5e-18 &\cr
        %Ebit E10* & 1620 & 18 & \si{\tera\hertz} & 3.2e-18 & \cr % https://en.bitcoin.it/wiki/Mining_hardware_comparison
        \bottomrule
    \end{tabular}
    \caption{}
    \label{table:sha256}
\end{table*}

\subsection{Analytical considerations}
% TODO attacker model

\dots

First, we formulate $r(n)$ as the expected relative cost \emph{per bit payload} with respect to prefix length $n$ from above equation (\ref{eq:totalcost}), fixing parameters $N$, $c$, $f$, and consider the continuation $r\colon \mathbb{R}_+ \to \mathbb{R}$ onto the reals, that is
\[ 
    r(n) = \left(\frac{2c2^{n}}{p^2+p} + fp\right)/N = \frac{2c 2^{n} n^2}{N^2 (N+n)}+\frac{f}{n}. 
\]
Thus function $r$ constitutes a continuous real-valued function, and we can find optimal $n^*$ yielding minimum cost per bit payload using usual derivative test.
Since $N\gg n\gg 2$, we approximate
\begin{align*}
    \frac{\partial}{\partial n} r(n)   &\approx \frac{\partial}{\partial n} \left( \frac{2c 2^{n} n^2}{N^3}+\frac{f}{n} \right)\\
                                         &= \frac{2c 2^{n} n (n \log (2)+2)}{N^3}-\frac{f}{n^2}\\
                                         &\approx \left(\frac{2 \log (2)\,c}{N^3}\right) 2^{n} n^2 -\frac{f}{n^{2}}.
\end{align*}
Solving for $\partial/\partial n\, r(n^*)=0$, we get minimum cost at $n=n^*$, that is, with $\alpha, \beta \ll 1$ constants, $W$ the \emph{Lambert $W$ function},
\[ r(n^*) \approx f/n^* = \frac{\alpha\cdot f}{W\left(\beta\cdot\sqrt[4]{f N^3/c}\right)}. \]
is the approximative optimal relative cost, depending on $f$, $N$, $c$.
Assuming $W\in \mathcal{O}(\log)$, we observe
\begin{enumerate}
    \item relative cost is effectively linear with respect to transaction fee $f$,
    \item halvening the computation cost allows for a constant number of additional payload bits to be included, having same fixed budget; $1/r(n^*) \in \mathcal{O}(-\log(c))$;
    \item even though total cost $N\cdot r(n^*)$ of data is asymtotically sublinear in $\mathcal{O}(N / \log(N))$, for any reasonable quantity $N$ total cost grows effectively linear.
\end{enumerate}

\section{Cost of malicious attack}

In this section, we address the problem of data inclusion from the perspective of a malicious adversary.
As was outlined in the introduction, deliberate inclusion of problematic data into the Blockchain, such as copyright-protected, illegal, or politically sensitive content, could render the possession of the Blockchain illegal in most jurisdictions.
Hence, due to persistency of the blockchain, this malicious inclusion harms every participant in Bitcoin's network, and might make participation even legally impossible.

% TODO motivation, scale of resources availiable to the attacker(?)
Since Bitcoin's concensus mechanism is the basis for its security, we assume a consensus attack (i.e. \enquote*{51\%-attack} – controling more than honest miner's hashrate) as an upper bound for a disruptive attack against Bitcoin's network.
Therefore, we evaluate the cost of such an attack including malicious content \emph{in terms of computational power}.
Moreover, it seems adequate to measure the required computational power \emph{relative to the concensus attack}.

As we have already seen, data inclusion using {P2SH} outputs via brute-forcing many redeem scripts is the computationally least expensive method.
To compare the computational cost to Bitcoin's mining process, we again rely on the previously stated assumption, that the expense of  hashing a single {P2SH} transaction is not greater than hashing a single block during the Proof-of-Work mining process.

% TODO argue for bandwith limit imposed by block frequency (1 TX per block = 3117 outputs per block = 3117*n bits per 10 minutes)

\begin{table*}
    \centering
    \sisetup{round-mode=figures,table-figures-exponent=1,table-number-alignment=center,scientific-notation=engineering,round-precision=3,exponent-to-prefix=true}
    \begin{tabular}{S[table-figures-exponent=0]S[table-figures-exponent=0]S[table-figures-exponent=0]SS}
        \toprule
        {Bit} & {max. data rate (\si{\byte\per\second})} & {per block (\si{\kilo\byte})} & {req. hashrate (\si{\Hz})} & {rel. to Network}\\
        \midrule
 20 & 12.9875 & 7.7925 & 5447352. & 0.0000000000000518795 \\
 25 & 16.2344 & 9.74063 & 174315274. & 0.00000000000166015 \\
 30 & 19.4813 & 11.6888 & 5578088776. & 0.0000000000531247 \\
 35 & 22.7281 & 13.6369 & 178498840822. & 0.00000000169999 \\
 40 & 25.975 & 15.585 & 5711962906296. & 0.0000000543996 \\
 45 & 29.2219 & 17.5331 & 182782813001482. & 0.00000174079 \\
 50 & 32.4688 & 19.4813 & 5849050016047432. & 0.0000557052 \\
 55 & 35.7156 & 21.4294 & 187169600513517800. & 0.00178257 \\
 60 & 38.9625 & 23.3775 & 5989427216432571000. & 0.0570422 \\
        \bottomrule
     \end{tabular}
    \caption{}
\end{table*}

\section{Discussion and Conclusion}

\end{document}
